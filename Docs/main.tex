\documentclass[a4paper, 10pt]{article}

% Use packages here
\usepackage[a4paper, left=1cm, right=1cm, top=1.5cm, bottom=2cm]{geometry}
\usepackage{blindtext} % This package is used for testing
\usepackage[british]{babel} % This package is used for localisation
\usepackage{paralist} % This package creates compact lists
\usepackage{multicol} % This package arranges contents in columns
\usepackage{indentfirst} % This package is used for indentation in paragraphs
\usepackage{array}
\usepackage{amsmath} % This package is used for conditional math equations
\usepackage{graphicx} % This package is used for inserting and manipulating on images
\usepackage{titlesec} % For section, subsection and subsubsection and so on.

% Preamble here
\setlength\columnsep{20pt} % This is the default columnsep for all pages

\setcounter{secnumdepth}{4}

\graphicspath{ {../Diagrams/} {../Diagrams/KeyExpansion/} }

\hyphenation{
    trans-form-ation 
    exponent-ially 
    crypto-graphic
    crypto-graphy 
    sub-titu-tion 
    ta-ble 
    rou-tine 
    rectang-ular 
    expan-sion 
    secur-ity 
    encryp-tion 
    decryp-tion 
    specifi-cation 
    inter-mediate 
    multi-pli-cation 
    poly-nomials 
    compris-ing 
    stand-ard
}

% Document begins here
\begin{document}

    %%%%%%%%%
    % TITLE %
    %%%%%%%%%
    \title{\centering{IMPLEMENTING AES-256 ON FPGA}}
    \author{}
    \maketitle % Call this command to make title

    \begin{center}
        Ho Hai Cong Thuan\\
        Student, Computer Engineering, University of Information Technology, HCMC, Vietnam\\
        hohaicongthuan@gmail.com
    \end{center}
    \begin{multicols}{2}
        
        %%%%%%%%%%%%
        % ABSTRACT %
        %%%%%%%%%%%%
        \textbf{\textit{Abstract\textemdash}In the modern days, the amount of data grow exponentially, including classified and sensitive data that need to be kept secured. For this reason, many cryptographic techniques have been invented for the purpose. AES is one of them. It provides fast and secure data encryption which are the reasons this algorithm is chosen for this project.}

        \textbf{The goal of this project is to implement a fully functional AES encryption and decryption system using 256-bit key on FPGA.}
        
        \bigskip

        \textbf{\textit{Keywords\textemdash}AES-256; cryptography; data security; FPGA; encryption; decryption.}

        %%%%%%%%%%%
        % CONTENT %
        %%%%%%%%%%%
        \section{INTRODUCTION}
            \textit{The Advanced Encryption Standard (AES)}, also known as \textit{Rijndael} is a specification for encrypting electronic data first introduced by the \textit{U.S. National Institute of Standards and Technology (NIST)} in 2001. It provides a fast and secure way to encrypt data and uses symmetric keys encryption which means both the encryption and decryption processes using the same key. The key length for AES could be 128, 192 and 256 bits. This paper concentrates on AES using 256-bit key which will be refered to as AES-256 for the rest of this paper.
            
            \subsection{Concepts used in AES-256}
                % \begingroup
                    % \setlength{\tabcolsep}{10pt} % Default value: 6pt
                    % \renewcommand{\arraystretch}{1.5} % Default value: 1
                    \noindent
                    \begin{tabular}{m{2cm} m{5cm}}
                        \textbf{\textit{Key expansion}} & Routine used to generate a series of Round Keys from the Cipher Key.\\
                        \hline
                        \textbf{\textit{State}} & Intermediate Cipher result that can be pictured as a rectangular array
                        of bytes, having four rows and \textbf{\textit{Nb}} columns.\\
                        \hline
                        \textbf{\textit{S-box}} & Non-linear substitution table used in several byte substitution transformation and in the Key Expansion routine to perform a one-for-one substitution of a byte value.\\
                        \hline
                        \textbf{\textit{Word}} & A group of 32 bits that is treated either as a single entity or as an array of 4 bytes.\\
                    \end{tabular}
                % \endgroup
            
            \subsection{Abbreviations and Symbols used in AES-256}
                % \begingroup
                    \setlength{\tabcolsep}{5pt} % Default value: 6pt
                    \renewcommand{\arraystretch}{1.5} % Default value: 1
                    \noindent
                        \begin{tabular}{m{1cm} m{6.5cm}}
                            \textbf{\textit{Nb}} & Number of columns (32-bit words) comprising the State. For this
                            standard, \textbf{\textit{Nb}} = 4.\\
                            \hline
                            \textbf{\textit{Nk}} & Number of 32-bit words comprising the Cipher Key. For this standard, \textbf{\textit{Nk}} = 8.\\
                            \hline
                            \textbf{\textit{Nr}} & Number of rounds, which is a function of \textbf{\textit{Nk}} and \textbf{\textit{Nb}} (which is
                            fixed). For this standard, \textbf{\textit{Nr}} = 14.\\
                            \hline
                            XOR & Exclusive-OR operation\\
                            \hline
                            $\oplus$ & Exclusive-OR operation\\
                            \hline
                            $\otimes$ & Multiplication of two polynomials (each with degree $<$ 4) modulo $x^{4}+1$\\
                            \hline
                            $\bullet$ & Finite field multiplication\\
                        \end{tabular}
                % \endgroup

        \section{AES-256}
            \subsection{Key Expansion}
                The Key Expansion routine in AES-256 takes a 256-bit cipher key and generate a set of $Nb(Nr+1)$ (which is 60) words. These words are smaller parts that make up round keys, each round key has four words. These round keys involve in the \textit{Add Round Key} in the encryption and decryption process.

                There are 3 functions that participate in the key scheduling process:

                % \begingroup
                    \setlength{\tabcolsep}{5pt} % Default value: 6pt
                    \renewcommand{\arraystretch}{1.5} % Default value: 1
                    \noindent
                        \begin{tabular}{m{2cm} m{6cm}}
                            \textbf{\textit{RotWord}} & Takes a four-byte word $[a_{0},a_{1},a_{2},a_{3}]$ and performs rotation one byte to the left and returns $[a_{1},a_{2},a_{3},a_{0}]$.\\
                            \hline
                            \textbf{\textit{SubWord}} & Takes a four-byte word and substitute each byte with the corresponding byte in the S-box.\\
                            \hline
                            \textbf{\textit{Rcon}} & The round constants, which is given in the form $[rc_{i},00_{16},00_{16},00_{16}]$ with $i$ starts from 1. $rc_{i}$ is defined as in (1)\\
                        \end{tabular}
                    \end{multicols}

                    \noindent
                    \begin{figure}[t]
                        \centering
                        \includegraphics[width=.7\textwidth]{S-Box.png}
                        \caption{Substitution Box (S-Box) used in SubBytes transformation.}
                        \label{fig:SBox}
                    \end{figure}

                    \begin{multicols}{2}
                        \begin{equation}
                            rc_{i}=
                            \begin{cases}
                                1 & \text{if } i=1\\
                                2 \cdot  rc_{i-1} & \text{if } i>1 \text{ and } rc_{i-1}<80_{16}\\
                                \left(2 \cdot  rc_{i-1}\right)\oplus 11\text{B}_{16} & \text{if } i>1 \text{ and } rc_{i-1} \ge 80_{16}
                            \end{cases}
                        \end{equation}
                % \endgroup

            \subsubsection{Algorithm}
                In AES-256, the first two round keys (first 8 words) are filled with the cipher key. For the rest, $w[i]$ word is generated using $w[i-1]$ word.

                Loop through the following steps until we have generated $Nb(Nr+1)$ words.

                If $i$ is divisible by \textbf{\textit{Nk}}, $w[i-1]$ is rotated by the function \textbf{\textit{RotWord}} and then substituted by the function \textbf{\textit{SubWord}}. The final result is \textbf{\textit{XOR}}-ed with \textbf{\textit{Rcon[i/Nk]}} and assigned to $w[i]$. Otherwise, if $i$ dividing by \textbf{\textit{Nk}} results in $4$ as the remainder, only \textbf{\textit{SubWord}} is performed on $w[i-1]$.

                $w[i]$ will then be \textbf{\textit{XOR}}-ed with $w[i-Nk]$ and the result is assigned back to itself. $i$ is incremented by 1.

                After finishing the algorithm, a set of $Nb(Nr+1)$ words is generated. Round Keys are created by grouping four words each sequentially. At this point, we have one round key for the initial round and 14 round keys for 14 rounds during the encryption or decryption processes, with the total of 15 round keys.

            \subsection{Bytes Substitution}

            Bytes Substitution transformation is denoted by \textit{SubBytes} function. This independently replaces all the bytes in the \textit{State} with the corresponding bytes using a \textit{substituion box} (or \textit{S-Box}), which is shown in the figure \ref{fig:SBox}.

            The higher 4 bits determine the coordinate of the row and lower 4 bits determine the coordinate of the column. For example, ...

            \subsection{Shift Rows}

            In the \textit{ShiftRows} transformation, the last three row of the \textit{State} will be rotated to the left with different byte offsets.

            The first row of the \textit{State} is unaffected. The second row will be rotated to the left by one byte, two bytes for the third row, and three bytes for the fourth row. For instance, ...

            \subsection{Mix Columns}

            \textit{MixColumns} transformation operates column-by-column on the \textit{State}. Each column is multiplied with a fixed matrix which results in a new column with new values.

            The fixed matrix used in \textit{MixColumns} is shown below:

            \begin{equation}
            \begin{bmatrix}
                02 & 03 & 01 & 01\\
                01 & 02 & 03 & 01\\
                01 & 01 & 02 & 03\\
                03 & 01 & 01 & 02
            \end{bmatrix}
        \end{equation}

            \subsection{Add Round Keys}

            Round Keys generated from \textit{Key Expansion} rountine will be used in this transformation. Round Key is added to the \textit{State} by a bitwise XOR operation with the corresponding bytes.

        \section{IMPLEMENTATION}

            \subsection{Key Expansion}

            The design of the Key Expansion routine is shown in figure \ref{fig:KeyExpansion}. As in the design, the total of 15 round keys, including round key for the initial round, will be created first before they could involve in the \textbf{\textit{Add Round Key}} function in the encryption and decryption process.
            
            This circuit will generate words that make up the Round Key one-by-one and store them in the \textit{Register File}. The \textit{Register File} has the total amount of 64 registers but only 60 of them are used because we only need to generate 60 words.

            The module \textit{Rcon} and \textit{SubWord} are implemented in the form of look-up tables (LUTs). A \textit{Counter} will keep track of which word the circuit is currently generating. After done generating all words, the counter will stop itself until there is a \textit{Reset} signal and \textit{ready} signal is set indicating all round keys have been generated. Each round key can be retrieved by inputting the appropriate address (from 0 to 14) of that round key. Note that round keys can only be correctly retrieved when the \textit{ready} signal is set.

            \subsection{Bytes Substitution}

            Text here.

            \subsection{Shift Rows}

            Text here.

            \subsection{Mix Columns}

            Text here.

            \subsection{Add Round Keys}

            Text here.

        \section{RESULTS}

        (N/A)

        \section{CONCLUSION}

        (N/A)

        %%%%%%%%%%%%%%%%%%%%%%%%%%
        % REFERENCE SECTION HERE %
        %%%%%%%%%%%%%%%%%%%%%%%%%%
        \begin{thebibliography}{3}
            \bibitem{AES}
            Federal Information Processing Standard (FIPS) 197.
            \textit{Advanced Encryption Standard (AES)}
            26 November 2001.

            \bibitem{sam}
            Sam Trenholme.
            \textit{Rijndael's key schedule.\\https://samiam.org/key-schedule.html}.

            \bibitem{understand_mixcol}
            Kit Choy Xintong.
            \textit{Understanding AES Mix-Columns Transformation Calculation}.
        \end{thebibliography}

    \end{multicols}

\end{document}
